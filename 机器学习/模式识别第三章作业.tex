\documentclass{article}
\usepackage{CJKutf8}
\usepackage{graphicx}
\usepackage{subfigure}
\usepackage{caption}
\usepackage{amssymb}
\usepackage{titlesec}
\usepackage{fancyhdr}
\usepackage{geometry}
\usepackage{amsmath}
\usepackage{listings}
\usepackage{color}

\definecolor{dkgreen}{rgb}{0,0.6,0}
\definecolor{gray}{rgb}{0.5,0.5,0.5}
\definecolor{mauve}{rgb}{0.58,0,0.82}

\lstset{ %
  language=Octave,                % the language of the code
  basicstyle=\footnotesize,           % the size of the fonts that are used for the code
  numbers=left,                   % where to put the line-numbers
  numberstyle=\tiny\color{gray},  % the style that is used for the line-numbers
  stepnumber=2,                   % the step between two line-numbers. If it's 1, each line 
                                  % will be numbered
  numbersep=5pt,                  % how far the line-numbers are from the code
  backgroundcolor=\color{white},      % choose the background color. You must add \usepackage{color}
  showspaces=false,               % show spaces adding particular underscores
  showstringspaces=false,         % underline spaces within strings
  showtabs=false,                 % show tabs within strings adding particular underscores
  frame=single,                   % adds a frame around the code
  rulecolor=\color{black},        % if not set, the frame-color may be changed on line-breaks within not-black text (e.g. commens (green here))
  tabsize=2,                      % sets default tabsize to 2 spaces
  captionpos=b,                   % sets the caption-position to bottom
  breaklines=true,                % sets automatic line breaking
  breakatwhitespace=false,        % sets if automatic breaks should only happen at whitespace
  title=\lstname,                   % show the filename of files included with \lstinputlisting;
                                  % also try caption instead of title
  keywordstyle=\color{blue},          % keyword style
  commentstyle=\color{dkgreen},       % comment style
  stringstyle=\color{mauve},         % string literal style
  escapeinside={\%*}{*)},            % if you want to add LaTeX within your code
  morekeywords={*,...}               % if you want to add more keywords to the set
}

\newcommand{\trimat}[3]{
    $$
        \left(\begin{matrix}
        #1 \\
        #2 \\
        #3 \\
        \end{matrix}\right)
    $$
}

\begin{document}

\begin{CJK*}{UTF8}{gbsn}

    \begin{center}
        \textsc{\LARGE 第三章模式识别作业}\\[1.5cm]
        \textsc{\LARGE 李佳政 201828013229075}\\[1.cm]
    \end{center}

    %\noindent LaTeX
    % 在需要缩进的地方设置\setlength\parindent{2em}   
    \setlength\parindent{0em}   
    \section{判别函数数目}

    因为共3类单独满足多数情况1,也就是说明通过线性判别可以完全分开这三类与剩余7类。3类共需要3个判别函数,额外1个判别函数用来分开另外7类。7类满足多数情况2,所以共需要$7\times(7-1)/2=21$,所以总共需要$3+1+21=25$个判别函数。
    
    \section{三个多分类情况}
    % h 水平排列

    \begin{figure}[htbp]
        \centering
            \subfigure[多类情况1]{
                \begin{minipage}[t]{0.3\linewidth}
                \centering
                \includegraphics[width=2in]{pic1.png}
                \end{minipage}
            }
            \subfigure[多类情况2]{
                \begin{minipage}[t]{0.3\linewidth}
                \centering
                \includegraphics[width=2in]{pic2.png}
                \end{minipage}
            }
            \subfigure[多类情况3]{
                \begin{minipage}[t]{0.3\linewidth}
                \centering
                \includegraphics[width=2in]{pic3.png}
                \end{minipage}
            }
        \end{figure}


    %\includegraphics[height=2.4in,width=0.5\textwidth]{pic1.png}
    %\includegraphics[height=2.4in,width=0.5\textwidth]{pic2.png}
    %\includegraphics[height=2.4in,width=0.5\textwidth,height=0.5\textwidth]{pic3.png}


    \section{判别函数组合}
    \subsection{} 
    线性可分的情况下,共需要n+1个权向量(增广矩阵),即4个系数分量。
    \subsection{}
    二次判别函数共需要
    \begin{math}
        N = \frac{(r+n)!}{r!} \times n ! = \frac{5!}{2!3!} = 10
    \end{math}
    
    \section{二分类感知器}
    将$\omega_{2}$的训练样本乘以-1,得到增广向量。\\
    \begin{math}
        x_{1} = [0, 0, 0, 1], x_{2} = [1, 0, 0, 1], x_{3} = [1, 0, 1, 1], x_{4} = [1, 1, 0, 1] \\
        x_{5} = [0, 0, -1, -1], x_{6} = [0, -1, -1, -1], x_{7} = [0, -1, 0, -1], x_{8} = [-1, -1, -1, -1]
    \end{math}

    \noindent
    迭代学习率$C=1$,设置初始权重系数$w_{1}=[0, 0, 0, 0]^{T}$
    然后开始迭代,\\
    \begin{math}
        1) w(1)^{T}x_{1}=0 \ngtr 0, w(2) = w(1) + x_{1} = [0, 0, 0, 1] \\
        2) w(2)^{T}x_{2}=1 > 0, w(3) = w(2) = [0, 0, 0, 1] \\
    \end{math}
    到第四轮都正确分类
    ...
    权重变化如下: \\
    第1轮迭代,权重为,[0. 0. 0. 1.]\\
    第2轮迭代,权重为,[0. 0. 0. 1.]\\
    第3轮迭代,权重为,[0. 0. 0. 1.]\\
    第4轮迭代,权重为,[0. 0. 0. 1.]\\
    第5轮迭代,权重为,[ 0.  0. -1.  0.]\\
    第6轮迭代,权重为,[ 0.  0. -1.  0.]\\
    第7轮迭代,权重为,[ 0. -1. -1. -1.]\\
    第8轮迭代,权重为,[ 0. -1. -1. -1.]\\
    第9轮迭代,权重为,[ 0. -1. -1.  0.]\\
    第10轮迭代,权重为,[ 1. -1. -1.  1.]\\
    第11轮迭代,权重为,[ 1. -1. -1.  1.]\\
    第12轮迭代,权重为,[ 1. -1. -1.  1.]\\
    第13轮迭代,权重为,[ 1. -1. -2.  0.]\\
    第14轮迭代,权重为,[ 1. -1. -2.  0.]\\
    第15轮迭代,权重为,[ 1. -1. -2.  0.]\\
    第16轮迭代,权重为,[ 1. -1. -2.  0.]\\
    第17轮迭代,权重为,[ 1. -1. -2.  1.]\\
    第18轮迭代,权重为,[ 1. -1. -2.  1.]\\
    第19轮迭代,权重为,[ 2. -1. -1.  2.]\\
    第20轮迭代,权重为,[ 2. -1. -1.  2.]\\
    第21轮迭代,权重为,[ 2. -1. -2.  1.]\\
    第22轮迭代,权重为,[ 2. -1. -2.  1.]\\
    第23轮迭代,权重为,[ 2. -2. -2.  0.]\\
    第24轮迭代,权重为,[ 2. -2. -2.  0.]\\
    第25轮迭代,权重为,[ 2. -2. -2.  1.]\\
    第26轮迭代,权重为,[ 2. -2. -2.  1.]\\
    第27轮迭代,权重为,[ 2. -2. -2.  1.]\\
    第28轮迭代,权重为,[ 2. -2. -2.  1.]\\
    第29轮迭代,权重为,[ 2. -2. -2.  1.]\\
    第30轮迭代,权重为,[ 2. -2. -2.  1.]\\
    第31轮迭代,权重为,[ 2. -2. -2.  1.]\\
    第32轮迭代,权重为,[ 2. -2. -2.  1.]\\
    第33轮迭代,权重为,[ 2. -2. -2.  1.]\\
    第34轮迭代,权重为,[ 2. -2. -2.  1.]\\
    第35轮迭代,权重为,[ 2. -2. -2.  1.]\\
    第36轮迭代,权重为,[ 2. -2. -2.  1.]\\
    第37轮迭代,权重为,[ 2. -2. -2.  1.]\\
    第38轮迭代,权重为,[ 2. -2. -2.  1.]\\
    第39轮迭代,权重为,[ 2. -2. -2.  1.]\\
    第40轮迭代,权重为,[ 2. -2. -2.  1.]\\
    \textbf{程序如下所示}:
    \lstset{language=python}
    \begin{lstlisting}
import numpy as np
xs = np.array([
    [0, 0, 0, 1],
    [1, 0, 0, 1],
    [1, 0, 1, 1],
    [1, 1, 0, 1],
    
    [0, 0, -1, -1],
    [0, -1, -1, -1],
    [0, -1, 0, -1],
    [-1, -1, -1, -1],
])

w = np.zeros((xs.shape[1], 1))

for i in range(100):
    err = False
    for j, x in enumerate(xs, 1):
        if np.matmul(w.T, x) <= 0:
            w += x.reshape(-1, 1)
            err = True
    if not err: break
    \end{lstlisting}

    \section{多分类感知器}
        将模式样本写成增广形式,即
    \begin{math}
        x_{1}=$$\trimat{-1}{-1}{0}$$
            $,$
        x_{2} = $$\trimat{0}{0}{1}$$
        $,$
        x_{3} = $$\trimat{1}{1}{1}$$
    \end{math}
    \\
    取初始值为
        $w_{1}=w_{2}=w_{3}=\trimat{0}{0}{0}$
    \\
    1)  第一次迭代,以$x_{1}$为样本,
        \begin{math}
        d_{1}(x_{1})=d_{2}(x_{1})=d_{3}(x_{1})=0
        \end{math}.
        因为$d_{1}(x_{1}) \ngtr d_{2}(x_{1})$
        且$d_{1}(x_{1}) \ngtr d_{3}(x_{1})$,
        所以更新权重得到:\\
        \begin{math}
            w_{1}(2)=w_{1}(1) + x_{1} = $$\trimat{-1}{-1}{1}$$
            $,$ \\
            w_{2}(2)=w_{2}(1) - x_{1} = $$\trimat{1}{1}{-1}$$
            $,$ \\
            w_{3}(2)=w_{3}(1) - x_{1} = $$\trimat{1}{1}{-1}$$
        \end{math}

    \noindent
    2) 第二次迭代以$x_{2}$为样本,
        \begin{math}
            d_{1}(x_{2}) = 1, d_{2}(x_{2}) = -1, d_{3}(x{3}) = -1.
        \end{math}
        因为$d_{2}(x_{2}) \ngtr d_{1}(x_{2})$
        且$d_{2}(x_{2}) \ngtr d_{3}(x_{2})$,
        所以更新权重得到:\\
        \begin{math}
            w_{1}(3)=w_{1}(2) - x_{2} = $$\trimat{-1}{-1}{0}$$
            $,$ \\
            w_{2}(3)=w_{2}(2) + x_{2} = $$\trimat{1}{1}{0}$$
            $,$ \\
            w_{3}(3)=w_{3}(2) - x_{2} = $$\trimat{1}{1}{-2}$$
        \end{math}

    \noindent
    3) 第三次迭代以$x_{3}$为样本,
        \begin{math}
            d_{1}(x_{3}) = -2, d_{2}(x_{3}) = 2, d_{3}(x{3}) = 0.
        \end{math}
        因为$d_{3}(x_{3}) \ngtr d_{2}(x_{3})$,
        所以更新权重得到:\\
        \begin{math}
            w_{1}(4)=w_{1}(3) = $$\trimat{-1}{-1}{0}$$
            $,$ \\
            w_{2}(4)=w_{2}(3) - x_{3} = $$\trimat{0}{0}{-1}$$
            $,$ \\
            w_{3}(4)=w_{3}(3) + x_{3} = $$\trimat{2}{2}{-1}$$
        \end{math}
    
    \noindent
    4) 第四次迭代以$x_{1}$为样本,
        \begin{math}
            d_{1}(x_{1}) = 2, d_{2}(x_{1}) = -1, d_{3}(x{1}) = -5.
        \end{math}
        分类正确,更新权重得到:\\
        \begin{math}
            w_{1}(5)=w_{1}(4) = $$\trimat{-1}{-1}{0}$$
            $,$ \\
            w_{2}(5)=w_{2}(4) = $$\trimat{0}{0}{-1}$$
            $,$ \\
            w_{3}(5)=w_{3}(4) = $$\trimat{2}{2}{-1}$$
        \end{math}
    
    \noindent
    5) 第五次迭代以$x_{2}$为样本,
        \begin{math}
            d_{1}(x_{1}) = 0, d_{2}(x_{1}) = -1, d_{3}(x{1}) = -1.
        \end{math}
        因为$d_{2}(x_{2}) \ngtr d_{2}(x_{1})$,
        且$d_{2}(x_{2}) \ngtr d_{3}(x_{2})$,
        所以更新权重得到:\\
        \begin{math}
            w_{1}(6)=w_{1}(5) - x_{2} = $$\trimat{-1}{-1}{-1}$$
            $,$ \\
            w_{2}(6)=w_{2}(4) + x_{2} = $$\trimat{0}{0}{0}$$
            $,$ \\
            w_{3}(6)=w_{3}(4) - x_{2} = $$\trimat{2}{2}{-2}$$
        \end{math}

    \noindent
    6) 第六次迭代以$x_{3}$为样本,
    \begin{math}
        d_{1}(x_{3}) = -3, d_{2}(x_{3}) = 0, d_{3}(x{3}) = 2.
    \end{math}
    分类正确,更新权重得到:
    \begin{math}
        w_{1}(7)=w_{1}(6)$$ $,$  w_{2}(7)=w_{2}(6)$$ $,$  w_{3}(7)=w_{3}(6)$$
    \end{math}

    \noindent
    7) 第七次迭代以$x_{1}$为样本,
    \begin{math}
        d_{1}(x_{1}) = 1, d_{2}(x_{3}) = 0, d_{3}(x{3}) = -6.
    \end{math}
    分类正确,更新权重得到:
    \begin{math}
        w_{1}(8)=w_{1}(7)$$ $,$  w_{2}(8)=w_{2}(7)$$ $,$  w_{3}(8)=w_{3}(7)$$
    \end{math}

    \noindent
    8) 第八次迭代以$x_{2}$为样本,
    \begin{math}
        d_{1}(x_{2}) = -1, d_{2}(x_{2}) = 0, d_{3}(x{2}) = -2.
    \end{math}
    分类正确,更新权重得到:
    \begin{math}
        w_{1}(9)=w_{1}(8)$$ $,$  w_{2}(9)=w_{2}(8)$$ $,$  w_{3}(9)=w_{3}(8)$$
    \end{math}

    由于第6、7、8次迭代对三个样本均正确分类,所以结果向量
    \begin{math}
        $$w_{1}=\trimat{-1}{-1}{-1}, w_{2}=\trimat{0}{0}{0}, w_{3}=\trimat{2}{2}{-2}$$
    \end{math}
    三个判别函数为
    \begin{math}
        d_{1} = -x_{1}-x_{2}-1, d_{2}=0, d_{3} = 2x_{1}+2x_{2}-2
    \end{math}

    %% 六题
    \section{梯度法准则函数}
    准则函数的导数 \\
    \begin{math}
        \frac{\partial J}{\partial w} = \frac{1}{4 \left|| x \right||^{2}}[(w^{T}x-b)-|w^{T}x-b|]\times[x-x\times sgn(w^{T}x-b)],$其中$ 
        sgn(x)=
            \begin{cases}
                1& x > 0\\
                -1& x \leqslant 0 \\
            \end{cases}
    \end{math}  \\
    所以$w(k+1) = w(k) + C \times \frac{\partial J}{\partial w}$\\
    即分类函数迭代公式如下,
    \begin{equation*}
        w(k+1) = w(k) + C \times 
        \begin{cases}
            0   &   w^{T}x-b > 0    \\
            \frac{(b-w^{T}x)}{||x||^{2}}x  &   w^{T}x-b \leqslant 0
        \end{cases}
    \end{equation*}
    如果分类正确,则$w(k+1)=w(k)$

    %% 七题
    \section{二次势函数}
    $\omega_{1} = \{(0, 1)^{T}, (0, -1)^{T}\}$, $\omega_{2} = \{ (1, 0)^{T}, (-1, 0)^{T}\}$。
    
    二次Hermite多项式, 所以取前九项最低阶的二维的正交函数: \\
    $\varphi_{1}(x) = \varphi_{1}(x_{1}, x_{2}) = H_{0}(x_{1})H_{0}(x_{2}) = 1$ \\
    $\varphi_{2}(x) = \varphi_{2}(x_{1}, x_{2}) = H_{1}(x_{1})H_{0}(x_{2}) = 2x_{1}$ \\
    $\varphi_{3}(x) = \varphi_{3}(x_{1}, x_{2}) = H_{0}(x_{1})H_{1}(x_{2}) = 2x_{2}$ \\
    $\varphi_{4}(x) = \varphi_{4}(x_{1}, x_{2}) = H_{1}(x_{1})H_{1}(x_{2}) = 4x_{1}x_{2}$ 

    $\varphi_{5}(x) = \varphi_{5}(x_{1}, x_{2}) = H_{2}(x_{1})H_{0}(x_{2}) = 4x_{1}^{2}-2$ \\
    $\varphi_{6}(x) = \varphi_{6}(x_{1}, x_{2}) = H_{0}(x_{1})H_{2}(x_{2}) = 4x_{2}^{2}-2$ \\
    $\varphi_{7}(x) = \varphi_{7}(x_{1}, x_{2}) = H_{2}(x_{1})H_{1}(x_{2}) = 8x_{1}^{2}x_{2} - 4x_{2}$ \\
    $\varphi_{8}(x) = \varphi_{8}(x_{1}, x_{2}) = H_{1}(x_{1})H_{2}(x_{2}) = 8x_{2}^{2}x_{1} - 4x_{1}$ \\
    $\varphi_{9}(x) = \varphi_{9}(x_{1}, x_{2}) = H_{2}(x_{1})H_{2}(x_{2}) = (4x_{1}^{2}-2)(4x_{2}^{2}-2)$ \\
    按照第一类生成势函数:\\
    \begin{equation*}
    \begin{aligned}
        K(x, x_{k}) &= \sum_{i=1}^{9}\varphi_{i}(x)\varphi_{i}(x_{k}) \\
                    &= ...
    \end{aligned}
    \end{equation*}

    第一步: 取$x_{0}=(0, 1)^{T} \in \omega_{1}$,则$K_{1}(x) = K(x, x_{0}) = 1 + 4x_{2} -4(2x_{1}^{2}-1) + 4(2x_{2}^{2}-1) - 16(2x_{1}^{2}-1)(2x_{2}^{2}-1) - 16x_{2}(2x_{1}^{2}-1)$\\
    第二步:取$x_{1}=(0, -1)^{T} \in \omega_{1}$,则$K_{1}(x_{1})=5 > 0$,所以$K_{2}(x) = K_1(x)$\\
    第三步:取$x_{2}=(1, 0)^{T} \in \omega_{2}$,则$K_{2}(x_{2})=9 > 0$,所以$K_{3}(x) = K_{2}(x) - K(x, x_{2}) = 20x_{2}-20x_{1}-16x_{1}^{2}+16x_{2}^{2}-32x_{1}^{2}x_{2}+32x_{1}x_{2}^{2}$ \\
    第四步:取$x_{3}=(-1, 0)^{T} \in \omega_{2}$,则$K_{3}(x_{3})=4 > 0$,所以$K_{4}(x) = K_{3}(x) - K(x, x_{3}) = 15+20x_{2}-56x_{1}^{2}-8x_{2}^{2}-32x_{1}^{2}+64x_{1}^{2}x_{2}^{2}$ \\
    第五步:取$x_{0}=(0, 1)^{T} \in \omega_{1}$,则$K_{4}(x_{0}) = 27 > 0$,所以$K_5(x) = K_{4}(x)$ \\
    第六步:取$x_{1}=(0, -1)^{T} \in \omega_{1}$,则$K_{5}(x_{1})=-13 < 0$,所以$K_6(x) = K_5(x) + K{x, x_{6}} = -32x_{1}^{2} + 32x_{2}^{2}$. \\
    ...\\
    第七步到第十步模式都分类正确,所以最终的判别函数为$d(x)=-32x_{1}^{2}+32x_{2}^{2}$.

    \section{指数势函数}
    取$\alpha=1$,得到势函数$K(x, x_{k})=e^{-[(x_{1}-x_{k1})^{2} + (x_{2}-x_{k2})^{2}]}$,数据集与上一题相同,步骤如下:\\
    第一步:取$x_{0}=(0, 1)^{T} \in \omega_{1}$,则$K_{1}(x) = K(x, x_{0}) = e^{-(x_{1}^{2}+(x_{2}-1)^2)}$. \\
    第二步:取$x_{1}=(0, -1)^{T} \in \omega_{1}$,则$K_{1}(x_{1})=e^{-(0+4)} > 0$,所以$K_{2}(x) = K_{1}(x)$. \\
    第三步:取$x_{2}=(1, 0)^{T} \in \omega_{2}$,则$K_{2}(x_{2})=e^{-2} > 0$,所以$K_{3}(x) = K_{2}(x) - K_{2}(x, x_{2}) = e^{-(x_{1}^{2}+(x_{2}+1)^{2})} - e^{-[(x_{1}-1)^{2}+x_{2}^{2}]}$. \\
    第四步:取$x_{3}=(-1, 0)^{T} \in \omega_{2}$,则$K_{3}(x_{3})=e^{-2}-e^{-4} > 0$,所以$K_{4}(x) = K_{3}(x)-K(x, x_{3}) = e^{-(x_{1}^{2}+(x_{2}+1)^{2})} - e^{-[(x_{1}-1)^{2}+x_{2}^{2}]} - e^{[(x_{1}+1)^{2}+x_{2}^{2}]}$. \\
    第五步:取$x_{0}=(0, 1)^{T} \in \omega_{1}$,则$K_{4}(x_{0}) = 1-2e^{-2} > 0$,所以$K_{5}(x) = K_{4}(x)$. \\
    第六步:取$x_{1}=(0, -1)^{T} \in \omega_{1}$,则$K_{5}(x_{1})= e^{-4}-2e^{-2} < 0$,所以$K_6(x) = K_{5}(x) + K_{5}(x, x_{1}) = e^{-(x_{1}^{2}+(x_{2}+1)^{2})} - e^{-[(x_{1}-1)^{2}+x_{2}^{2}]} - e^{[(x_{1}+1)^{2}+x_{2}^{2}]} + e^{-[(x_{1}^{2} + (x_{2}+1)^{2})]}$. \\
    .... \\
    第七步到第十步模式都分类正确,所以最终的判别函数为
    $$d(x)=e^{-(x_{1}^{2}+(x_{2}+1)^{2})} - e^{-[(x_{1}-1)^{2}+x_{2}^{2}]} - e^{[(x_{1}+1)^{2}+x_{2}^{2}]} + e^{-[(x_{1}^{2} + (x_{2}+1)^{2})]}$$
    

\end{CJK*}
\end{document}
